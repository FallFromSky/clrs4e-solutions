Let $C$ be the $n\times n$ matrix with $c_{ij}=1$, and 0s elsewhere.
Observe that $CA$ is the matrix, in which the $i$th row is the $j$th row of $A$, while other rows consist of 0s.
Similarly, $BC$ is the matrix, in which the $j$th column is the $i$th column of $B$, while other columns consist of 0s.
Let $D=I+C$ and $D'=I-C$.
Then, $A'=DA$ and $B'=BD'$.

Observe, that since $i\ne j$, for any $1\le p$, $q\le n$ the terms $c_{pk}$ and $c_{kq}$ can't be both nonzero.
Thus, $\sum_{k=1}^nc_{pk}c_{kq}=0$, and so $CC$ is a zero matrix.
Then,
\begin{align*}
    DD' &= (I+C)(I-C) \\
    &= II+CI-IC-CC \\
    &= I+C-C-CC \\
    &= I,
\end{align*}
which means that $D'=D^{-1}$.

Returning to the matrices $A'$ and $B'$, we get
\begin{align*}
    A'B' &= DABD' \\
    &= D(AB)D^{-1} \\
    &= (DI)D^{-1} \\
    &= DD^{-1} \\
    &= I,
\end{align*}
hence $B'=(A')^{-1}$.
