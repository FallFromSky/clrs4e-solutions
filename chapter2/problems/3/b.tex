The following procedure takes the same set of parameters as \proc{Horner} and evaluates $P(x)$ using the naive method.

\begin{codebox}
\Procname{$\proc{Polynomial-Evaluate}(A,n,x)$}
\li $p\gets0$
\li \For $i\gets0$ \To $n$
\li     \Do $s\gets A[i]$
\li         \For $j\gets1$ \To $i$ \label{li:polynomial-evaluate-inner-for-begin}
\li             \Do $s\gets s\cdot x$
                \End \label{li:polynomial-evaluate-inner-for-end}
\li         $p\gets p+s$
        \End
\li \Return $p$
\end{codebox}

The \kw{for} loop of lines \ref{li:polynomial-evaluate-inner-for-begin}--\ref{li:polynomial-evaluate-inner-for-end} iterates $i$ times for each $i=0$, 1, \dots, $n$, for a total of $\sum_{i=0}^ni=n(n+1)/2$ iterations.
The running time of \proc{Polynomial-Evaluate} is therefore $\Theta(n^2)$, so the algorithm is less efficient as compared to \proc{Horner}.
