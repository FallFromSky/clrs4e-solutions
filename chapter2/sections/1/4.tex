The description leads to the following pseudocode:

\begin{codebox}
\Procname{$\proc{Linear-Search}(A,n,x)$}
\li \For $i\gets1$ \To $n$ \label{li:linear-search-for-begin}
\li     \Do \If $A[i]\isequal x$
\li         \Then \Return $i$
            \End
        \End \label{li:linear-search-for-end}
\li \Return \nil
\end{codebox}

We'll prove the loop invariant:
\begin{quote}
    At the start of each iteration of the \kw{for} loop of lines \ref{li:linear-search-for-begin}--\ref{li:linear-search-for-end}, the value $x$ does not appear in the subarray $A[1\subarr i-1]$.
\end{quote}

\begin{description}
    \item[Initialization:] Prior to the first iteration of the loop, $i=1$.
    Then, the subarray $A[1\subarr i-1]$ is empty and as such does not contain $x$.
    \item[Maintenance:] Suppose that just before the $i$th iteration the value $x$ does not appear in the subarray $A[1\subarr i-1]$.
    The loop will terminate in this iteration, if $A[i]$ equals $x$.
    Otherwise, we increment $i$ and move on to the next iteration, so the loop invariant remains true.
    \item[Termination:] The loop terminates as soon as it encounters $i$ such that $A[i]$ equals $x$, or when it reaches $i=n+1$.
    In the first case the procedure returns the index $i$.
    In the second case it returns \nil, which is correct, since by the loop invariant, the value $x$ does not appear in the subarray $A[1\subarr n]$.
\end{description}
