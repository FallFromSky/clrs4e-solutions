\begin{codebox}
\Procname{$\proc{Selection-Sort}(A,n)$}
\li \For $i\gets1$ \To $n-1$ \label{li:selection-sort-for-begin}
\li     \Do $\id{min}\gets i$ \label{li:selection-sort-min-init}
\li         \For $j\gets i+1$ \To $n$
\li             \Do \If $A[j]<A[\id{min}]$
\li                     \Then $\id{min}\gets j$ \label{li:selection-sort-min-update}
                        \End
                \End
\li         exchange $A[\id{min}]$ with $A[i]$
        \End \label{li:selection-sort-for-end}
\end{codebox}

The outer \kw{for} loop maintains the following invariant:
\begin{quote}
    Just prior to the $i$th iteration of the \kw{for} loop of lines \ref{li:selection-sort-for-begin}--\ref{li:selection-sort-for-end}, the subarray $A[1\subarr i-1]$ contains the $i-1$ smallest elements of array $A$ in sorted order.
\end{quote}
According to the invariant, once the loop terminates after performing the $n-1$ iterations, the subarray $A[1\subarr n-1]$ contains the $n-1$ smallest elements of $A$ in sorted order.
The last element then must be greater than or equal to each element of the subarray $A[1\subarr n-1]$, which means that the $n-1$ iterations suffice to sort the entire array.

In the $i$th iteration of the outer \kw{for} loop we determine the smallest element of $A[i\subarr n]$.
After initializing the position of a potential minimum in line \ref{li:selection-sort-min-init}, we need to examine the remaining $n-i$ positions in the inner \kw{for} loop.
Let's assign statement costs for each line of the algorithm, so that line $k$ has cost $c_k$.
For each $i=1$, 2, \dots, $n-1$, let $t_i$ denote the number of times line \ref{li:selection-sort-min-update} is executed during the $i$th iteration of the outer \kw{for} loop.
Then, the running time of the algorithm on an $n$-element array is
\[
    T(n) = c_1n+c_2(n-1)+c_3\sum_{i=1}^{n-1}(n-i+1)+c_4\sum_{i=1}^{n-1}(n-i)+c_5\sum_{i=1}^{n-1}t_i+c_6(n-1).
\]
In the worst case $t_i=n-i$ for each $i=1$, 2, \dots, $n-1$, and by the fact that
\begin{align*}
    \sum_{i=1}^{n-1}(n-i+1) &= \sum_{i=2}^ni \\[1mm]
    &= \frac{n(n+1)}{2}-1
\end{align*}
and
\begin{align*}
    \sum_{i=1}^{n-1}(n-i) &= \sum_{i=1}^{n-1}i \\[1mm]
    &= \frac{n(n-1)}{2},
\end{align*}
the running time of \proc{Selection-Sort} is
\begin{align*}
    T(n) &= c_1n+c_2(n-1)+c_3\left(\frac{n(n+1)}{2}-1\right)+c_4\left(\frac{n(n-1)}{2}\right) \\
    &\qquad {}+c_5\left(\frac{n(n-1)}{2}\right)+c_6(n-1) \\[1mm]
    &= \left(\frac{c_3}{2}+\frac{c_4}{2}+\frac{c_5}{2}\right)n^2\!+\left(c_1+c_2+\frac{c_3}{2}-\frac{c_4}{2}-\frac{c_5}{2}+c_6\right)n-(c_2+c_3+c_6).
\end{align*}
Therefore, $T(n)$ can be expressed as $an^2+bn+c$ for constants $a$, $b$, and $c$ that depend on the statement costs $c_k$, and so $T(n)=\Theta(n^2)$.

The best case differs from the worst case in the values $t_i$, which all are 0.
But this difference only cancels out the terms involving $c_5$ in the above formula, and so it doesn't affect the rate of growth of the running time of the algorithm for the best case.
