\exercise
Let $h:\mathbb{Z}\to\mathbb{Z}\times\mathbb{Z}$ be the function we are looking for.
Since the inverse of a bijection is also a bijection, we will focus on finding the inverse $h^{-1}:\mathbb{Z}\times\mathbb{Z}\to\mathbb{Z}$, which is equivalent to finding a way to sequentially number each ordered pair with integer elements, so that each integer is used as the number of some pair.
We will now describe the construction of one of such mappings.

Let us make some simplification\dash instead of numbering pairs by integers, we will restrict ourselves to natural numbers.
Let $g:\mathbb{Z}\times\mathbb{Z}\to\mathbb{N}$ and $f:\mathbb{N}\to\mathbb{Z}$ be such bijections that $h^{-1}=f\circ g$.
It was already noted on page 1162 that $f(n)=(-1)^n\lceil n/2\rceil$ is bijective, so all that remains is to find the function $g$.

Consider the numbering of ordered pairs with integer elements shown in \refFigure{B.3-4} in the form of a spiral.
Since each such pair $(x,y)$ is assigned a unique natural number, we can treat this spiral as a description of the function $g$.
\begin{figure}[htb]
    \subimport{./}{4.tikz}
    \caption{A bijection from $\mathbb{Z}\times\mathbb{Z}$ to $\mathbb{Z}$.
    Individual natural numbers denote the values of this bijection for points with integer coordinates in the Cartesian coordinate system.
    } \label{fig:B.3-4}
\end{figure}
Let $d=\max(|x|,|y|)$ and $D=(2d-1)^2-1$.
Informally, $d$ and $D$ denote, respectively, the number of the ``lap'' around the origin covered by the spiral at the time of passing through $(x,y)$, and the largest number on the spiral during the previous ``lap''.
Then we can define the function $g$ as follows:
\[
    g(x,y) =
    \begin{Cases}
        0 & \text{if $d=|x|=|y|=0$}, \\
        D+\phantom{1}d+y & \text{if $d=x\ne|y|$}, \\
        D+3d-x & \text{if $d=y\ne0$}, \\
        D+5d-y & \text{if $d=-x\ne|y|$}, \\
        D+7d+x & \text{if $d=-y\ne0$}.
    \end{Cases}
\]
