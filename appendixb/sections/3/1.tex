\exercise
\subexercise
For any function $f:A\to B$, $f(A)\subseteq B$, which results directly from the definition of the range of $f$\!.
Additionally, if $f$ is injective, then each argument $a\in A$ has a distinct image $f(a)\in f(A)$, so $|A|=|f(A)|$.
Hence, $|A|\le|B|$.

\subexercise
For any function $f:A\to B$, where $A$ is finite, it holds $|A|\ge|f(A)|$, because there may be arguments $a_1$, $a_2\in A$ such that $f(a_1)=f(a_2)$.
Additionally, if $f$ is surjective, then $f(A)=B$, and therefore $|A|\ge|B|$.
