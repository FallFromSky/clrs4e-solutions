\exercise
We proceed by induction on $n$.
The base case is when $n=0$.
The only full binary tree that has no internal nodes is a one-node tree, for which we have $e=i=0$ and the statement holds.

Now let $n\ge1$ and suppose that the statement holds for all full binary trees with less than $n$ internal nodes.
We can construct an arbitrary full binary tree $T$ with $n$ internal nodes by choosing any pair of full binary trees, whose numbers of internal nodes sum up to $n-1$, for its left subtree $T_L$ and its right subtree $T_R$.
For $k\ge0$, if $T_L$ has $k$ internal nodes, $T_R$ has $n-k-1$ internal nodes.
By the conclusion made in \refExercise{B.5-3}, $T_L$ and $T_R$ have $k+1$ and $n-k$ leaves, respectively.

Let $i_L$ and $i_R$ be the internal path lengths of $T_L$ and $T_R$, respectively, and similarly, let $e_L$ and $e_R$ be their external path lengths.
The depth of each node in $T$\dash other than the root, for which the depth is 0\dash is 1 more than the depth of this node in either $T_L$ or $T_R$.
Hence,
\begin{align*}
    i &= 0+(i_L+k)+(i_R+n-k-1) \\
    &= i_L+i_R+n-1
\end{align*}
and
\begin{align*}
    e &= (e_L+k+1)+(e_R+n-k) \\
    &= e_L+e_R+n+1.
\end{align*}
By the inductive hypothesis, $e_L=i_L+2k$ and $e_R=i_R+2(n-k-1)$, so
\begin{align*}
    e-i &= e_L+e_R+n+1-i_L-i_R-n+1 \\
    &= i_L+2k+i_R+2(n-k-1)-i_L-i_R+2 \\
    &= 2n.
\end{align*}
Therefore, the statement holds for all full binary trees.
