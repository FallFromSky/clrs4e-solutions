Symmetry and transitivity are defined using implications.
For an implication ``$p$ implies $q$'' to hold true, it isn’t required for $p$ to be true.
A relation on a set remains symmetric and transitive, if there is an element in the set, that is not related to any element in that set.
Reflexivity, on the other hand, requires that each element is related to itself.
Thus, there exist relations that are symmetric and transitive relations but not reflexive (an example of one is given in \refExercise{B.2-3}(c)).
