For any $k\ge2$, if graph $G$ is $k$-colorable, but is not $(k-1)$-colorable, then for every two different colors used in a $k$-coloring of $G$ there are two adjacent vertices, one with the first color and the other with the second color.
If that weren't true, there would be two different colors that we could not distinguish, and so we could reduce the number of colors required, obtaining a $(k-1)$-coloring of $G$.

Based on the above reasoning, we have $\binom{k}{2}\le|E|$.
We use the fact that for $k\ge2$ it holds that $k/2\le k-1$, and hence
\begin{align*}
    k^2 &= 4\cdot\frac{k}{2}\cdot\frac{k}{2} \\[2mm]
    &\le 4\cdot\frac{k(k-1)}{2} \\
    &= 4\binom{k}{2} \\[1mm]
    &\le 4|E| \\
    &= O(|V|).
\end{align*}
Therefore, $k=O(\sqrt{|V|})$.
