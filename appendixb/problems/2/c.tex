\subproblem
\begin{theorem}
    For any undirected graph $G=(V,E)$ the set $V$ can be partitioned into two subsets such that for any vertex $v\in V$ at least half of its neighbors do not belong to the subset that $v$ belongs to.
\end{theorem}

\begin{proof}
    Let us divide the vertices of $G$ arbitrarily into two disjoint subsets $V_1$ and $V_2$.
    Let
    \[
        S = \{(v_1,v_2)\in E:v_1\in V_1,\,v_2\in V_2\}.
    \]
    If more than half of the neighbors of some vertex $v\in V_1$ are also in $V_1$, then moving $v$ to $V_2$ increases the size of set $S$.
    Similarly, for vertices $v\in V_2$ with this property\dash by moving them to set $V_1$, we increase set $S$.
    This process must terminate after a finite number of steps, because $S$ cannot grow indefinitely.
    The obtained partition $V=V_1\cup V_2$ satisfies the desired property.
\end{proof}
