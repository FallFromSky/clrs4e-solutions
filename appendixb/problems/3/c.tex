Assume that $n\ge6$.
We'll remove one edge from the original tree to cut off a subtree of at most $\lfloor n/2\rfloor$ vertices.
Consider any of the tree's branches $\langle x_0$, $x_1$, \dots, $x_k\rangle$, as well as the function $s$, both defined in the solution to part (a).
There is an edge $(x_j,x_{j+1})$, such that $s(x_j)>\lfloor n/2\rfloor$ and $s(x_{j+1})\le\lfloor n/2\rfloor$.
From inequality \eqref{eq:subtree-sizes-on-branch} we have
\begin{align*}
    s(x_{j+1}) &\ge (s(x_j)-1)/2 \\
    &> (\lfloor n/2\rfloor-1)/2 \\
    &\ge \lfloor n/2\rfloor/3.
\end{align*}
We cut off the subtree rooted at $x_{j+1}$ and continue partitioning the remaining tree, with at most
\[
    \lfloor n/2\rfloor-s(x_{j+1}) < (2/3)\lfloor n/2\rfloor
\]
vertices still to remove from it.
In each subsequent step the number of nodes remaining to be cut is reduced by a factor of $2/3$.
Therefore, we'll need to remove $O(\log_{3/2}n)=O(\lg n)$ edges.
