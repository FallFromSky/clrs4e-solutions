We'll assume that $n$ is an exact power of 2.
The procedure \proc{Matrix-Add-Recursive} takes three $n\times n$ matrices $A$, $B$ and $C$, and computes $C=C+(A+B)$.
Like \proc{Matrix-Multiply-Recursive}, in the divide step it partitions the matrices as in equation (4.2).
But in the conquer step it adds each submatrix of $A$ to the corresponding submatrix of $B$ and adds the sum to the corresponding submatrix of $C$:
\begin{align*}
    C_{11} &= C_{11}+A_{11}+B_{11}, \\
    C_{12} &= C_{12}+A_{12}+B_{12}, \\
    C_{21} &= C_{21}+A_{21}+B_{21}, \\
    C_{22} &= C_{22}+A_{22}+B_{22},
\end{align*}
which involves four recursive calls.

\begin{codebox}
\Procname{$\proc{Matrix-Add-Recursive}(A,B,C,n)$}
\li \If $n\isequal1$
\li     \Then \Comment{Base case.}
\li         $c_{11}\gets c_{11}+a_{11}+b_{11}$
\li         \Return
        \End
\li \Comment{Divide.}
\li partition $A$, $B$, and $C$ into $n/2\times n/2$ submatrices
    \Indentmore
\zi     $A_{11}$, $A_{12}$, $A_{21}$, $A_{22}$; $B_{11}$, $B_{12}$, $B_{21}$, $B_{22}$;
\zi     and $C_{11}$, $C_{12}$, $C_{21}$, $C_{22}$; respectively
    \End \label{li:matrix-add-recursive-partition}
\li \Comment{Conquer.}
\li $\proc{Matrix-Add-Recursive}(A_{11},B_{11},C_{11},n/2)$ \label{li:matrix-add-recursive-recursive-begin}
\li $\proc{Matrix-Add-Recursive}(A_{12},B_{12},C_{12},n/2)$
\li $\proc{Matrix-Add-Recursive}(A_{21},B_{21},C_{21},n/2)$
\li $\proc{Matrix-Add-Recursive}(A_{22},B_{22},C_{22},n/2)$ \label{li:matrix-add-recursive-recursive-end}
\end{codebox}

Let $T(n)$ be the worst-case time required by this algorithm to add two $n\times n$ matrices.
Each recursive call in lines \ref{li:matrix-add-recursive-recursive-begin}--\ref{li:matrix-add-recursive-recursive-end} adds two $n/2\times n/2$ matrices contributing $T(n/2)$ to the overall running time, so all four recursive calls take $4T(n/2)$ time.
If we assume that the algorithm uses index calculations to partition the matrices in line \ref{li:matrix-add-recursive-partition}, the recurrence for its running time is
\[
    T(n) = 4T(n/2)+\Theta(1).
\]
Applying case~1 of the master theorem, we obtain the solution $T(n)=\Theta(n^2)$.
However, if the matrices are partitioned by copying, then the recurrence changes to
\[
    T(n) = 4T(n/2)+\Theta(n^2),
\]
which, by case~2 of the master theorem, has the solution $T(n)=\Theta(n^2\lg n)$.
