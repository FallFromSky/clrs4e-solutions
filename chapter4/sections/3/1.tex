\subexercise
Let's adopt the inductive hypothesis that $T(n)\le cn^2$ for all $n\ge n_0$, where $c$, $n_0>0$ are constants.
Assume by induction that this bound holds for all numbers at least as big as $n_0$ and less than $n$.
In particular, if $n\ge n_0+1$, it holds for $n-1$, yielding $T(n-1)\le c(n-1)^2$.
Substituting into the recurrence yields
\begin{align*}
    T(n) &\le c(n-1)^2+n \\
    &= cn^2-2cn+c+n \\
    &= cn^2+(c-1/2)(1-2n)+1/2 \\
    &\le cn^2,
\end{align*}
where the last step holds as long as $(c-1/2)(1-2n)+1/2\le0$.
If we further constrain $n_0$ to be no less than 1 (so that $n\ge2$), we can satisfy this inequality by choosing any $c\ge2/3$.

We've shown that the inductive hypothesis holds for the inductive case, and now we'll establish it for the base case, i.e., when $n_0\le n<n_0+1$.
Let's pick $n_0=1$.
By our convention, $T(n)$ is algorithmic, so $T(1)$ is constant, and we can choose $c$ large enough to satisfy the inequality $T(1)\le c\cdot1^2$.

Therefore, $T(n)\le cn^2$ for all $n\ge1$, which implies that the recurrence has the solution $T(n)=O(n^2)$.

\subexercise
We adopt the inductive hypothesis that $T(n)\le c\lg n$ for all $n\ge n_0$, where $c$, $n_0>0$ are constants.
Let $n\ge2n_0$ and assume by induction that $T(n/2)\le c\lg(n/2)$.
Then,
\begin{align*}
    T(n) &\le c\lg(n/2)+\Theta(1) \\
    &= c\lg n-c\lg2+\Theta(1) \\
    &= c\lg n-c+\Theta(1) \\
    &\le c\lg n,
\end{align*}
where the last step holds if we constrain the constant $c$ to be sufficiently large that for $n\ge2n_0$, $c$ dominates the anonymous constant hidden by the $\Theta(1)$ term.

Now let's pick $n_0=2$, so that $\lg n>0$ for all $n\ge n_0$.
Both $T(2)$ and $T(3)$ are constants, so we can choose $c$ large enough to satisfy the inequalities $T(2)\le c\lg2$ and $T(3)\le c\lg3$, establishing the inductive hypothesis for the base cases, when $n_0\le n<2n_0$.

We've proven that $T(n)\le c\lg n$ for all $n\ge2$, so $T(n)=O(\lg n)$.

\subexercise
We'll prove each bound separately, starting from the lower bound.
Consider the inductive hypothesis that $T(n)\ge c_1n\lg n$ for all $n\ge n_0$, where $c_1$, $n_0>0$ are constants.
Let $n\ge2n_0$ and assume by induction that $T(n/2)\ge c_1(n/2)\lg(n/2)$.
Then,
\begin{align*}
    T(n) &\ge 2c_1(n/2)\lg(n/2)+n \\
    &= c_1n\lg n-c_1n+n \\
    &\ge c_1n\lg n && \text{(as long as $c_1\le1$)}.
\end{align*}

Now it suffices to show that the bound holds when $n_0\le n<2n_0$.
Let's pick $n_0=2$, so that $\lg n>0$ for all $n\ge n_0$.
Both $T(2)$ and $T(3)$ are constants, so we can choose $c_1$ small enough (but positive) to satisfy the inequalities $T(2)\ge c_1(2\lg2)$ and $T(3)\ge c_1(3\lg3)$, establishing the inductive hypothesis for the base cases.

We proceed similarly for the upper bound, by considering the inductive hypothesis $T(n)\le c_2n\lg n$ for all $n\ge n_0$, where $c_2>0$ is another constant.
If we let $n\ge2n_0$ and assume that $T(n/2)\le c_2(n/2)\lg(n/2)$, we obtain
\begin{align*}
    T(n) &\le 2c_2(n/2)\lg(n/2)+n \\
    &= c_2n\lg n-c_2n+n \\
    &\le c_2n\lg n && \text{(as long as $c_2\ge1$)}.
\end{align*}
Now, for $n_0=2$ (which we fixed earlier) we only need to verify that the bound holds when $n=2$ or $n=3$.
Of course is does, because both inequalities $T(2)\le c_2(2\lg2)$ and $T(3)\le c_2(3\lg3)$ can be satisfied by a sufficiently large $c_2$.

We've proven that $c_1n\lg n\le T(n)\le c_2n\lg n$ for all $n\ge2$, so $T(n)=\Theta(n\lg n)$.

\subexercise
We adopt the inductive hypothesis that $T(n)\le c(n-34)\lg(n-34)$ for all $n\ge n_0$, where $c>0$, $n_0>34$ are constants.
Let $n\ge2n_0-34$ and assume by induction that $T(n/2+17)\le c(n/2+17-34)\lg(n/2+17-34)$.
Then,
\begin{align*}
    T(n) &\le 2c(n/2+17-34)\lg(n/2+17-34)+n \\
    &= c(n-34)\lg(n/2-17)+n \\
    &= c(n-34)\lg(n-34)-c(n-34)+n \\
    &\le c(n-34)\lg(n-34).
\end{align*}
The last step holds as long as $c(n-34)\ge n$, or $c\ge n/(n-34)$.
If $n\ge35$, then $n/(n-34)\le35$, so we can pick any $c\ge35$.

Now let $n_0\le n<2n_0-34$.
The expression $c(n-34)\lg(n-34)$ is positive for all $n>35$, so let's pick $n_0=36$.
We satisfy the inequalities $T(36)\le c(2\lg2)$ and $T(37)\le c(3\lg3)$ by further constraining $c$ to be sufficiently large, establishing the inductive hypothesis for the base cases.

Given the fact that the function $f(n)=n\lg n$ defined over $n\ge1$ is strictly increasing, we have in particular that for $n>34$, $f(n-34)<f(n)$.
By this observation and by the upper bound shown above, we get that for all $n\ge36$,
\begin{align*}
    T(n) &\le c(n-34)\lg(n-34) \\
    &< cn\lg n,
\end{align*}
so $T(n)=O(n\lg n)$.

\subexercise
In the proof of the lower bound we adopt the inductive hypothesis that $T(n)\ge c_1n$ for all $n\ge n_0$, where $c_1$, $n_0>0$ are constants.
Let $n\ge3n_0$ and assume by induction that $T(n/3)\ge c_1(n/3)$.
Then,
\begin{align*}
    T(n) &\ge 2c_1(n/3)+\Theta(n) \\
    &= c_1n-(c_1/3)n+\Theta(n) \\
    &\ge c_1n,
\end{align*}
where the last step holds if we let $c_1$ be small enough that for $n\ge3n_0$, the anonymous function hidden by the $\Theta(n)$ term dominates the quantity $(c_1/3)n$.

Now let $n_0\le n<3n_0$.
If we pick $n_0=1$, we only need to prove the bound for $n=1$ and $n=2$.
By further decreasing $c_1$ we can satisfy both $T(1)\ge c_1\cdot1$ and $T(2)\ge c_1\cdot2$, establishing the inductive hypothesis for the base cases.

We can show the upper bound in a similar manner.
Our inductive hypothesis is that $T(n)\le c_2n$ for all $n\ge n_0$, where $c_2>0$ is another constant.
Let $n\ge3n_0$ and assume that $T(n/3)\le c_2(n/3)$.
We have
\begin{align*}
    T(n) &\le 2c_2(n/3)+\Theta(n) \\
    &= c_2n-(c_2/3)n+\Theta(n) \\
    &\le c_2n,
\end{align*}
where for the last step to hold we must choose $c_2$ large enough, so that for all $n\ge3n_0$, the quantity $(c_2/3)n$ dominates the anonymous function hidden by the $\Theta(n)$ term.

With $n_0=1$ let's increase $c_2$ appropriately, so that inequalities $T(1)\le c_2\cdot1$ and $T(2)\le c_2\cdot2$ hold, which proves the bound for the base cases.

We've proven that $c_1n\le T(n)\le c_2n$ for all $n\ge1$, so $T(n)=\Theta(n)$.

\subexercise
To show that $T(n)=\Omega(n^2)$ we adopt the inductive hypothesis that $T(n)\ge cn^2$ for all $n\ge n_0$, where $c$, $n_0>0$ are constants.
By letting $n\ge2n_0$ and assuming that $T(n)\ge c(n/2)^2$, we get
\begin{align*}
    T(n) &\ge 4c(n/2)^2+\Theta(n) \\
    &= cn^2+\Theta(n) \\
    &\ge cn^2.
\end{align*}
The last step holds if we pick $n_0$ such that the function hidden by the $\Theta(n)$ term is nonnegative for all $n\ge n_0$.

Now let $n_0\le n<2n_0$.
By our convention about algorithmic recurrences, all such $T(n)$ are constants, so we can satisfy the inequalities $T(n)\ge cn^2$ for all $n_0\le n<2n_0$ by finding a suitable value for $c$.

The upper bound of $O(n^2)$ for this recurrence is shown in \refExercise{4.3-2}, therefore $T(n)=\Theta(n^2)$.
