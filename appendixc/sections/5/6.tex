\begin{lemma}
    Let $\alpha$, $p$, and $q$ be nonnegative reals, such that $p+q=1$.
    Then,
    \begin{equation} \label{eq:alpha}
        pe^{\alpha q}+qe^{-\alpha p} \le e^{\alpha^2\!/2}.
    \end{equation}
\end{lemma}

\begin{proof}
    Consider the function $f(\alpha)=e^{\alpha^2\!/2}-(pe^{\alpha q}+qe^{-\alpha p})$ for $\alpha\ge0$, and examine its derivatives:
    \begin{align*}
        f'(\alpha) &= \alpha e^{\alpha^2\!/2}-pq(e^{\alpha q}-e^{-\alpha p}), \\
        f''(\alpha) &= \alpha^2e^{\alpha^2\!/2}+e^{\alpha^2\!/2}-pq(qe^{\alpha q}+pe^{-\alpha p}).
    \end{align*}
    It's true that $f''(0)=1-pq\ge0$ and we'll show that $f''(\alpha)>0$ for all $\alpha>0$.
    Since
    \begin{align*}
        pq-1/4 &= p(1-p)-1/4 \\
        &= -p^2+p-1/4 \\
        &= -(p-1/2)^2 \\
        &\le 0
    \end{align*}
    or, equivalently, since $pq\le1/4$,
    \begin{align*}
        pq(qe^{\alpha q}+pe^{-\alpha p}) &\le (qe^{\alpha q}+pe^{-\alpha p})/4 \\
        &\le (e^{\alpha q}+e^{-\alpha p})/4 \\
        &= e^{-\alpha p}(e^{\alpha q+\alpha p}+1)/4 \\
        &= e^{-\alpha p}(e^\alpha+1)/4 \\
        &\le (e^\alpha+1)/4.
    \end{align*}
    Combining this inequality with $\alpha^2e^{\alpha^2\!/2}>0$, that holds for all $\alpha>0$, we get
    \[
        f''(\alpha) > e^{\alpha^2\!/2}-(e^\alpha+1)/4.
    \]
    Now it suffices to show that $4e^{\alpha^2\!/2}\ge e^\alpha+1$ or, since $e^{\alpha^2\!/2}>1$, that
    \[
        3e^{\alpha^2\!/2} \ge e^\alpha.
    \]
    This is equivalent to $3e^{\alpha^2\!/2}\!/e^\alpha=3e^{\alpha^2\!/2-\alpha}\ge1$, which holds when
    \[
        \alpha^2\!/2-\alpha+\ln3 \ge 0.
    \]
    The left-hand side expression attains the minimum at $\alpha=1$, equal to $\ln3-1/2>0$, which concludes the proof that $f''(\alpha)\ge0$ for $\alpha\ge0$.

    The fact we've just shown implies that the function $f'(\alpha)$ is monotonically increasing for all $\alpha\ge0$.
    Since $f'(0)=0$, it must be $f'(\alpha)\ge0$ for all $\alpha\ge0$, which in turn implies that the function $f(\alpha)$ is monotonically increasing.
    And since $f(0)=0$, it holds that $f(\alpha)\ge0$ for all $\alpha\ge0$, which completes the proof.
\end{proof}

The beginning of the main reasoning is based on the proof of Theorem C.8:
\begin{align*}
    \Pr{X-\mu\ge r} &= \Pr{e^{\alpha(X-\mu)}\ge e^{\alpha r}} \\
    &\le \E{e^{\alpha(X-\mu)}}e^{-\alpha r}.
\end{align*}
Then, with the same notations as in the original proof, it follows that
\[
    \E{e^{\alpha(X-\mu)}} = \prod_{i=1}^n\E{e^{\alpha(X_i-p_i)}}.
\]
Using inequality \eqref{eq:alpha} for $p=p_i$ and $q=q_i$, we get
\begin{align*}
    \E{e^{\alpha(X_i-p_i)}} &= e^{\alpha(1-p_i)}p_i+e^{\alpha(0-p_i)}q_i \\
    &= p_ie^{\alpha q_i}+q_ie^{-\alpha p_i} \\
    &\le e^{\alpha^2\!/2},
\end{align*}
and so
\begin{align*}
    \E{e^{\alpha(X-\mu)}} &= \prod_{i=1}^n\E{e^{\alpha(X_i-p_i)}} \\
    &\le \prod_{i=1}^ne^{\alpha^2\!/2} \\
    &= \exp(\alpha^2n/2).
\end{align*}
Therefore,
\begin{align*}
    \Pr{X-\mu\ge r} &\le \E{e^{\alpha(X-\mu)}}e^{-\alpha r} \\
    &\le \exp(\alpha^2n/2-\alpha r).
\end{align*}
Now we need to choose the value of $\alpha$ that minimizes the last expression.
The argument of the exponential function in that expression is a quadratic function with respect to $\alpha$, so it's easy to show that it attains a minimum at $\alpha=r/n$.
We finally get
\begin{align*}
    \Pr{X-\mu\ge r} &\le \exp\left((r/n)^2n/2-(r/n)r\right) \\
    &= e^{-r^2\!/{2n}}.
\end{align*}
