Intuitively, the situation when each flip in a series results in a head is less likely when the series consists of twice more flips, while we expect the same number of heads as before.
Let's prove our intuition.
Let $X_m$ for $m\ge0$ be the random variable denoting the number of heads obtained in $m$ flips of a fair coin.
When $n\le m$, $\Pr{X_m=n}=b(n;m,1/2)$.
Thus, the ratio of both studied probabilities is
\begin{align*}
    \frac{\Pr{X_{2n}=n}}{\Pr{X_n=n}} &= \frac{\binom{2n}{n}\left(\frac{1}{2}\right)^{2n}}{\binom{n}{n}\left(\frac{1}{2}\right)^n} \\[1mm]
    &= \frac{\binom{2n}{n}}{2^n} \\[1mm]
    &= \frac{2^n}{\sqrt{\pi n}}\,(1+O(1/n)) && \text{(by equation (C.11))} \\
    &> 1,
\end{align*}
which confirms our guess.
