We will only count non-empty $k$-substrings, so $k\ge1$.
The first $k$-substring occupies positions 1 through $k$ in the $n$-string, the second one occupies positions 2 through $k+1$, and so on.
The last $k$-substring ends at position $n$, so it must start at position $n-k+1$.
So an $n$-string has $n-k+1$ $k$-substrings in total.

By the rule of sum, the total number of all substrings of an $n$-string is
\begin{align*}
    \sum_{k=1}^n(n-k+1) &= \sum_{k=1}^nk \\[1mm]
    &= \frac{n(n+1)}{2} && \text{(by equation (A.1))} \\
    &= \binom{n+1}{2}.
\end{align*}
