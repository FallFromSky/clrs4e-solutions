\starred
We can enumerate the cases with $n=0$, 1, 2, 3 and $k=0$, 1, and confirm that the identity holds for all of them.

Now let $n\ge4$.
We have
\begin{align*}
    \binom{n}{k} &= \frac{n(n-1)\cdots(n-(k-1))}{k!} \\
    &= \frac{n^k}{k!}\left(1-\frac{1}{n}\right)\cdots\left(1-\frac{k-1}{n}\right) \\
    &\ge \frac{n^k}{k!}\left(1-\frac{k-1}{n}\right)^{k-1}.
\end{align*}
Now consider the function
\[
    f(x,y) = \left(1-\frac{x}{n}\right)^y,
\]
where $0\le x\le n$ and $y\ge0$.
Observe that the function $f$ monotonically decreases, when $x$ increases for a fixed $y$, or when $y$ increases for a fixed $x$.
Therefore, since $k-1<\sqrt{n}$,
\begin{align*}
    \left(1-\frac{k-1}{n}\right)^{k-1} &\ge \left(1-\frac{\sqrt{n}}{n}\right)^{k-1} \\
    &\ge \left(1-\frac{1}{\sqrt{n}}\right)^{\sqrt{n}}.
\end{align*}
It is true that
\[
    \left(1-\frac{1}{t}\right)^t \ge \frac{1}{4}
\]
for all $t\ge2$.
Therefore, when $n\ge4$,
\begin{align*}
    \binom{n}{k} &\ge \frac{n^k}{k!}\left(1-\frac{1}{\sqrt{n}}\right)^{\sqrt{n}} \\
    &\ge \frac{n^k}{4k!}.
\end{align*}
