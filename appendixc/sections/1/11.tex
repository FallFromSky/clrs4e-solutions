First, let's observe that
\begin{align*}
    (j+k)! &= j!\,(j+1)(j+2)\cdots k \\
    &\ge j!\,1\cdot2\cdots k \\
    &= j!\,k!,
\end{align*}
where equality holds only if $j=0$ or $k=0$.
Then
\begin{align*}
    \binom{n}{j+k} &= \frac{n!}{(j+k)!\,(n-j-k)!} \\
    &\le \frac{n!}{j!\,k!\,(n-j-k)!} \\[1mm]
    &= \frac{n!}{j!\,(n-j)!}\cdot\frac{(n-j)!}{k!\,(n-j-k)!} \\[1mm]
    &= \binom{n}{j}\binom{n-j}{k}.
\end{align*}

We can interpret the expression $\binom{n}{j+k}$ as the number of possible ways to choose $j+k$ items out of $n$, and the expression $\binom{n}{j}\binom{n-j}{k}$ as the number of ways to choose $j$ items out of $n$, and then $k$ items out of $n-j$ left after the first choice.
In each strategy we come up with a set of the $j+k$ selected items.
There is exactly one way to construct a given set of $j+k$ items using the first strategy, and at least one if we follow the second strategy.
It is so, because we can arbitrarily partition the set into the set of $j$ items that will be selected in the first step and the set of $k$ items that will be selected in the second step.
