\subexercise
By applying equation (3.17) twice, we get
\begin{align*}
    a^{\log_bc} &= (b^{\log_ba})^{\log_bc} \\
    &= (b^{\log_bc})^{\log_ba} \\
    &= c^{\log_ba}.
\end{align*}

\subexercise
We'll prove that
\[
    \lim_{n\to\infty}\frac{n!}{2^n} = \infty.
\]
Note that in the product
\[
    \frac{n!}{2^n} = \left(\frac{1}{2}\right)\left(\frac{2}{2}\right)\cdots\biggl(\frac{n}{2}\biggr)
\]
all the factors on the right-hand side are positive and, as $n$ increases, the final factors can grow arbitrarily large, so the product diverges to $\infty$.

Similarly, we'll prove that
\[
    \lim_{n\to\infty}\frac{n!}{n^n} = 0.
\]
We have
\[
    \frac{n!}{n^n} = \left(\frac{1}{n}\right)\left(\frac{2}{n}\right)\cdots\biggl(\frac{n}{n}\biggr).
\]
Each factor on the right-hand side is positive and does not exceed 1.
For $n$ large enough the initial factors can be arbitrarily close to 0, so the product approaches its limit of 0.

For the proof of equation (3.28), observe that since $n!\le n^n$, the upper bound on $\lg(n!)$ is $\lg n^n=n\lg n$.
We'll find an asymptotic lower bound by using Stirling's approximation.
Let $f(n)=1+\Theta(1/n)=\Theta(1)$ be the last factor on the right-hand side of formula (3.25).
Because the function $\!\sqrt{2\pi n}$ is strictly increasing, there exists a positive constant $n_0$ such that $\!\sqrt{2\pi n}f(n)\ge1$ for all $n\ge n_0$.
Also, we'll use the fact that $e<3$, so $e<\!\sqrt{n}$ when $n\ge9$.
For all $n\ge\max{n_0,9}$ we have
\begin{align*}
    \lg(n!) &= \lg\left(\!\sqrt{2\pi n}\left(\frac{n}{e}\right)^n\!f(n)\right) \\
    &\ge \lg\left(\frac{n}{e}\right)^n \\[1mm]
    &> \lg\left(\!\sqrt{n}\right)^n \\[1mm]
    &= \frac{n\lg n}{2}.
\end{align*}
Combining both upper and lower bounds, we obtain $\lg(n!)=\Theta(n\lg n)$.

\subexercise
Let $f(n)=\Theta(n)$.
By the definition of $\Theta$-notation, there exist positive constants $c_1$, $c_2$, and $n_0$ such that
\[
    c_1n \le f(n) \le c_2n
\]
for all $n\ge n_0$.
Taking logarithms of each side of the above inequalities yields
\begin{align*}
    \lg f(n) &\ge \lg(c_1n) \\
    &= \lg c_1+\lg n \\
    &\ge \lg\left(\frac{1}{\!\sqrt{n}}\right)+\lg n && \text{(as long as $n\ge1/c_1^2$)} \\
    &= \lg\!\sqrt{n} \\
    &= \frac{\lg n}{2}
\end{align*}
and
\begin{align*}
    \lg f(n) &\le \lg(c_2n) \\
    &= \lg c_2+\lg n \\
    &\le 2\lg n && \text{(as long as $n\ge c_2$)}.
\end{align*}
So, $\lg f(n)=\Theta(\lg n)$.
