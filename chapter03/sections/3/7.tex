\clarification{In its present form the exercise is trivial, since it asks to prove the statement that is almost exactly the definition of the golden ratio and its conjugate.
Below solution proves that they have the values (3.32) and (3.33), respectively.}

\noindent We have
\begin{align*}
    \left(\frac{1+\!\sqrt{5}}{2}\right)^2 &= \frac{1+2\!\sqrt{5}+5}{4} \\
    &= \frac{3+\!\sqrt{5}}{2} \\[1mm]
    &= \frac{1+\!\sqrt{5}}{2}+1
\end{align*}
and
\begin{align*}
    \left(\frac{1-\!\sqrt{5}}{2}\right)^2 &= \frac{1-2\!\sqrt{5}+5}{4} \\
    &= \frac{3-\!\sqrt{5}}{2} \\[1mm]
    &= \frac{1-\!\sqrt{5}}{2}+1,
\end{align*}
so both numbers $\frac{1+\!\sqrt{5}}{2}$ and $\frac{1-\!\sqrt{5}}{2}$ are the roots of the equation $x^2=x+1$, and therefore are the golden ratio $\phi$ and its conjugate $\widehat\phi$, respectively.
