We have
\begin{align*}
    \frac{\phi^0-\widehat\phi^0}{\sqrt{5}} &= 0 \\
    &= F_0
\end{align*}
and
\begin{align*}
    \frac{\phi^1-\widehat\phi^1}{\sqrt{5}} &= \frac{1+\sqrt{5}-\bigl(1-\sqrt{5}\bigr)}{2\sqrt{5}} \\
    &= 1 \\
    &= F_1,
\end{align*}
so the base case holds.
Now suppose that $i\ge2$.
The inductive hypothesis is that
\[
    F_{i-2} = \frac{\phi^{i-2}-\widehat\phi^{i-2}}{\sqrt{5}}
\]
and
\[
    F_{i-1} = \frac{\phi^{i-1}-\widehat\phi^{i-1}}{\sqrt{5}}.
\]
Then, we have
\begin{align*}
    F_i &= F_{i-2}+F_{i-1} \\[1mm]
    &= \frac{\phi^{i-2}-\widehat\phi^{i-2}}{\sqrt{5}}+\frac{\phi^{i-1}-\widehat\phi^{i-1}}{\sqrt{5}} \\[1mm]
    &= \frac{\phi^{i-2}(1+\phi)-\widehat\phi^{i-2}\bigl(1+\widehat\phi\bigr)}{\sqrt{5}} \\
    &= \frac{\phi^i-\widehat\phi^i}{\sqrt{5}} && \text{(by the definition of $\phi$ and $\widehat\phi$)},
\end{align*}
so the formula for $F_i$ holds for all $i\ge0$.
